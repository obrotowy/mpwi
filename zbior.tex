\documentclass{exam}
\usepackage{polski}
\usepackage{mathtools}
\usepackage{tabto}

\author{Emilian Zawrotny}
\title{Zbiór odpowiedzi do kolokwiów z MPwI}
\date{\today}
\begin{document}
    \maketitle
    \section{2024 Termin 1 grupa B}
    \begin{questions}
        \question iTrust
        \begin{parts}
            \part $P(\underline{T} = 0) = \frac{13}{20}$ \newline $P(\underline{T} = 1) = \frac{13}{40}$ \newline $P(\underline{T} = 2) = \frac{1}{40}$
            \part $V\underline{T} = 0.284375$
            \part $G_{\underline{T}}(z) = \frac{13}{20} + \frac{13}{40}z + \frac{1}{40}z^2$
        \end{parts}
        \question Wektory losowe
        \begin{parts}
            \part $k = -\frac{5}{6}$
            \part $p_{\underline{X}}(x) = 3x - \frac{5}{2}$ \newline
                $p_{\underline{Y}}(y) = -\frac{5}{3}y^2 + 2$
            \part Nie są niezależne statystycznie
            \part $corr(\underline{X}, \underline{Y}) = -\frac{9}{4}$
            \part $cov(\underline{X}, \underline{Y}) = \frac{15}{2}$
            \part Nie są ortogonalne, bo $corr(\underline{X}, \underline{Y}) \ne 0$. Nie są też nieskorelowane, bo $\lambda \ne 0$
        \end{parts}
    \end{questions}
    \section{2023 Termin 1 grupa 1}
    \begin{questions}
        \question Nadajniki i odbiorniki
        \begin{parts}
            \part $P(O_{01} | N_{00}) = 0.16$ \newline
                $P(O_{01} | N_{01}) = 0.72$ \newline
                $P(O_{01} | N_{10}) = 0.02$ \newline
                $P(O_{01} | N_{11}) = 0.09$
            \part $P(N_{01} | O_{01}) \approx 0.692$
        \end{parts}
        \question Zmienne losowe
        \begin{parts}
            \part $P(\underline{X} = k) = {4 \choose k} BER^k (1-BER)^{4-k}$
            \part $F(x) = \begin{cases}
                0 & \mbox{dla } x \le 0 \\
                0.6561 & \mbox{dla } x \in ( 0;1 \rangle \\
                0.9477 & \mbox{dla } x \in (1; 2\rangle \\
                0.9963 & \mbox{dla } x \in (2; 3\rangle \\
                0.9999 & \mbox{dla } x \in (3; 4\rangle \\
                1 & \mbox{dla } x \ge 4
            \end{cases}$
            \part $p(x) = \begin{cases}
                0.6561 & \mbox{dla } x = 0 \\
                0.2916 & \mbox{dla } x = 1 \\
                0.0486 & \mbox{dla } x = 2 \\
                0.0036 & \mbox{dla } x = 3 \\
                0.0001 & \mbox{dla } x = 4 \\
                0 & \mbox{dla pozostałych } x  
            \end{cases}$
            \part $P(\underline{X} \ge 2) = 0.0523$
            \part $P(1 \le \underline{X} < 2) = 0.2916$
        \end{parts}
    \end{questions}
    \section {2023 Termin 1 grupa 2}
    \begin{questions}
        \question iTrust
        \begin{parts}
            \part $P(\underline{X} \le 1) = \frac{4}{5}$
            \part $P(\underline{X} = 2) = \frac{1}{5}$
            \part Trzeba powtórzyć 2 razy, wtedy prawdopodobieństwo trafienia wynosi 96\%
        \end{parts}
    \end{questions}
\end{document}